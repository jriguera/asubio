
\thispagestyle{empty}

\textbf{Resumen} \\

El objetivo del presente proyecto es la creaci�n de una aplicaci�n de VoIP que permita a varios usuarios comunicarse entre s� usando ordenadores normales (PC's). El programa se ejecuta en plataformas Linux y, mediante una conexi�n bidireccional RTP/RTCP, permite mantener una conversaci�n entre dos o m�s personas a modo de tel�fono. \\

La aplicaci�n tiene una arquitectura abierta y modular, que permite extender sus funcionalidades con tres tipos de plugins: E/S de audio, effectos sobre audio y codecs de compresi�n de voz. El programa es multisesi�n, es decir, puede mantener varias conversaciones activas simult�neamente, con o sin redes multicast. Tambi�n est� totalmente preparado para soportar la IPv6, la pr�xima generaci�n de IP. \\

Ha sido desarrollado con Glib-2.0 lo que garantiza que el sistema es portable entre plataformas Unix e incluso -con peque�as modificaciones- a plataformas MS Windows. La GUI se ha desarrollado en GTK+2.0 y es totalmente intuitiva, permite mostrar una imagen del usuario con el que est� dialogando y dispone de una agenda en XML en la que gestionar los contactos: a�adir, borrar, ignorar llamadas, etc.

\vspace{1.5cm}

\textbf{Palabras clave} \\
\\
\texttt{VoIP, RTP/RTCP, IPv6, Glib/GTK+, multicast, multisesi�n, plugins, Linux}



\newpage\thispagestyle{empty}
$\ $
\newpage\thispagestyle{empty}
